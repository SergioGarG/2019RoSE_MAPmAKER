\documentclass[10pt,conference]{IEEEtran}
\IEEEoverridecommandlockouts

\pdfminorversion=4
%\overrideIEEEmargins
\usepackage{pbox}

\usepackage{subfig}
\usepackage{graphicx}

\usepackage{framed,lipsum}
\usepackage{microtype}
%\usepackage[font=footnotesize]{caption}
%\usepackage{float}
\usepackage{booktabs} % For formal tables
\usepackage{centernot}
\usepackage[Algorithmus]{algorithm}
\usepackage{algorithmic}
\usepackage{amsmath,amssymb,amsfonts}
\usepackage{balance}

\usepackage{graphicx}


\usepackage{array}
\usepackage{color}
\usepackage{listings}
%\usepackage[pdftex,bookmarks=false]{hyperref}
%\hypersetup{colorlinks=true,urlcolor=black,linkcolor=black,citecolor=black}


\newcommand \tab[1][1cm]{\hspace*{#1}}

%\usepackage{framed,lipsum}
%\usepackage{float}

\definecolor{codegreen}{rgb}{0,0.6,0}
\definecolor{codegray}{rgb}{0.5,0.5,0.5}
\definecolor{codepurple}{rgb}{0.58,0,0.82}
\definecolor{backcolour}{rgb}{0.95,0.95,0.92}
\definecolor{xtext}{RGB}{178, 14, 140}
%Code listing style named "mystyle"
\lstdefinestyle{mystyle}{
	commentstyle=\color{codegreen},
	keywordstyle=\color{magenta},
	numberstyle=\tiny\color{codegray},
	stringstyle=\color{codepurple},
	basicstyle=\scriptsize,
	breakatwhitespace=false,         
	breaklines=true,                 
	captionpos=b,                    
	keepspaces=true,                 
	numbers=none,                    
	numbersep=5pt,                  
	showspaces=false,                
	showstringspaces=false,
	showtabs=false,                  
	tabsize=2
}
%"mystyle" code listing set
\lstset{style=mystyle}

\usepackage{threeparttable}

\usepackage{url}
\usepackage{ltl}

%\usepackage{booktabs} % For formal tables
\usepackage{centernot}
\usepackage{algorithm}
\usepackage{algorithmic}
\usepackage{amsmath,amssymb,amsfonts}
%\usepackage{balance}
\let\labelindent\relax
\usepackage[inline]{enumitem}
\usepackage{rotating}
\usepackage{multirow}
\usepackage{tikz}
\usepackage{tabularx}
\usepackage{tabulary}
\usepackage{threeparttable}
\usepackage{rotating} %sidewaystable
\usepackage{microtype}
\usepackage{url}
\usepackage[normalem]{ulem}
\usepackage{chngpage}
\usepackage{relsize}
\usepackage{xspace}
%\usepackage{cite}
\usepackage{tcolorbox}

\newcommand{\totalrequirements}{208}
\newcommand{\notcoveredrequirements}{56}
\newcommand{\unbounded}{\ensuremath{\mathcal{N}}}
\newcommand{\bounded}{\ensuremath{\mathcal{T}}}
\newcommand{\setrobot}{\ensuremath{\mathcal{R}}}
\newcommand{\settasks}{\ensuremath{\mathcal{T}}}
\newcommand{\setpatterns}{\ensuremath{\mathcal{P}}}
\newcommand{\setareas}{\ensuremath{\mathcal{L}}}
\newcommand{\setareasOrdered}{\ensuremath{\mathcal{L_O}}}
\newcommand{\execute}{\ensuremath{\mathbb{E}}}
\newcommand{\query}{\ensuremath{\mathbb{Q}}}
\newcommand{\executeAll}{\ensuremath{\mathbb{E}_{All}}}
\newcommand{\setconditions}{\ensuremath{\mathcal{C}}}

\newcommand{\patternstoltl}{\ensuremath{\mathcal{PL}}}
\newcommand{\planner}{\ensuremath{\mathcal{J}}}
\newcommand{\execution}{\ensuremath{\mathcal{E}}}

\newcommand*\circled[1]{\tikz[baseline=(char.base)]{
		\node[shape=circle,draw,inner sep=2pt] (char) {#1};}}

\usepackage[colorinlistoftodos,prependcaption,textsize=tiny]{todonotes}

\newboolean{showcomments}
\setboolean{showcomments}{true} % toggle to show or hide comments
\ifthenelse{\boolean{showcomments}}
{\newcommand{\nb}[2]{
		\fcolorbox{black}{yellow}{\bfseries\sffamily\scriptsize#1}
		{\sf\small$\blacktriangleright$\textit{#2}$\blacktriangleleft$}
	}
	\newcommand{\version}{\emph{\scriptsize$-$working$-$}}
}
{\newcommand{\nb}[2]{}
	\newcommand{\version}{}
}
\newcommand\patrizio[1]{\nb{Patrizio}{#1}}
\newcommand\claudio[1]{\nb{Claudio}{#1}}
\newcommand\sergio[1]{\nb{Sergio}{#1}}
\newcommand\tb[1]{\nb{Thorsten}{#1}}
\newcommand\tomas[1]{\nb{Tomas}{#1}}

\usepackage{listings}
\colorlet{light-gray}{gray!20}
\newtheorem{remark}{Remark}
\newtheorem{problem}{Problem}
\usepackage[inline]{enumitem}


% correct bad hyphenation here
\hyphenation{op-tical net-works semi-conduc-tor}

% Macros for proof-reading
\usepackage[normalem]{ulem} % for \sout
\usepackage{xcolor}
\newcommand{\ra}{$\rightarrow$}
\newcommand{\ugh}[1]{\textcolor{red}{\uwave{#1}}} % please rephrase
\newcommand{\ins}[1]{\textcolor{blue}{\uline{#1}}} % please insert
\newcommand{\del}[1]{\textcolor{red}{\sout{#1}}} % please delete
\newcommand{\chg}[2]{\textcolor{red}{\sout{#1}}{\ra}\textcolor{blue}{\uline{#2}}} % please change

\setlength{\belowcaptionskip}{-20pt}
%\setlength{\textfloatsep}{1pt} 
%\setlength{\intextsep}{1pt}

\newcommand{\sera}{SERA\xspace}
\newcommand\parhead[1]{\vspace{.26mm}\noindent\textbf{{#1}}.}

\newcommand{\secref}[1]{Sec.\,\ref{#1}}
\newcommand{\Secref}[1]{Sec.\,\ref{#1}} %use at beginning of sentences (no abbrev. allowed)
\newcommand{\figref}[1]{Fig.\,\ref{#1}}
\newcommand{\Figref}[1]{Figure\,\ref{#1}} % use at beginning of sentence (no abbrev. allowed)
\newcommand{\tabref}[1]{Table\,\ref{#1}}
\newcommand{\lstref}[1]{Listing\,\ref{#1}}

\newcommand{\foot}[1]{\footnote{\url{#1}}}

\newcommand{\Cite}[1]{~\cite{#1}}

\newcommand{\toolName}{MAPmAKER}
\newcommand{\APs}{\mathbf{\Pi}}
\newcommand{\Lang}{\mathcal{L}} %language
\newcommand{\Set}{\mathsf{S}} %set
\newcommand{\Spec}{\mathbf{\Phi}}
\newcommand{\Epsilon}{\mathcal{E}}
\renewcommand{\i}{\iota}
\newcommand{\Nat}{\mathbb{N}} %natural numbers
\newcommand{\Real}{\mathbb{R}}
\newcommand{\Next}{\mathsf{X}}
\newcommand{\Until}{\mathsf{U}}
\newcommand{\Always}{\mathsf{G}}
\newcommand{\Event}{\mathsf{F}}
\newcommand{\false}{\mathit{false}}
\newcommand{\true}{\mathit{true}}
\newcommand{\trueval}{\ensuremath{\top}}
\newcommand{\falseval}{\ensuremath{\bot}}
\newcommand{\maybe}{\ensuremath{?}}
\renewcommand{\epsilon}{\varepsilon}
\newcommand{\prop}{\pi}
\newcommand{\ie}{{i.e., }}
\newcommand{\eg}{{e.g., }}
\newcommand{\progressive}{\varpi}
\newcommand{\move}{\mathit{move}}
\newcommand{\h}{h}
\renewcommand{\H}{H}
\newcommand{\parti}{\mathit{P}}
\newcommand{\Alpha}{\mathbf{\Sigma}}
\renewcommand{\mod}{\mathrm{\, mod \, }}
\newcommand{\suc}{\mathit{succ}}
\newcommand{\dist}{\mathrm{dist}}
\newcommand{\proj}{\mathrm{proj}}
\newcommand{\parspace}{\vskip 0.05in}


\usepackage{hyperref}
\hypersetup{
	colorlinks=true,
	linkcolor=black,
	filecolor=black,      
	urlcolor=blue,
	citecolor=black
}
\urlstyle{same}



\usepackage{cite}
\IEEEoverridecommandlockouts   


\begin{document}
	
	
	\title{MAPmAKER: Performing Multi-Robot LTL Planning Under Uncertainty}

	
	\author{\IEEEauthorblockN{Sergio Garc\'{i}a\IEEEauthorrefmark{1}, 
			Claudio Menghi\IEEEauthorrefmark{2}, and
			Patrizio Pelliccione\IEEEauthorrefmark{1}\IEEEauthorrefmark{3}
			}
		\IEEEauthorblockA{\IEEEauthorrefmark{1}Chalmers $|$ University of Gothenburg, 
			Gothenburg (Sweden)\\
			\IEEEauthorrefmark{2} University of Luxembourg, 
			Luxembourg City (Luxembourg)\\
			\IEEEauthorrefmark{3} University of L'Aquila, 
			L'Aquila (Italy)\\
			Email: sergio.garcia@gu.se, claudio.menghi@uni.lu, patrizio.pelliccione@gu.se}
	}	
	
	\maketitle

	\begin{abstract}
		Robot applications are being increasingly used in real life to help humans performing dangerous, heavy, and/or monotonous tasks.
		They usually rely on planners that given a robot or a team of robots compute plans that specify how the robot(s) can fulfill their missions.
		Current robot applications ask for planners that make automated planning 
		%\emph{tractable}	\claudio{To me maybe we can relax the term tractable, because our planner does not make the problem tractable.. there are planners that work much better than ours. I would say only that work in a decentralized fashion.}  
		possible even when only \emph{partial knowledge} about the environment in which the robots are deployed.
		To tackle such challenges we developed \toolName, %\chg{a proof of concepts application  that}
		which provides a decentralized planning solution  and  is able to work in partially known environments.
		Decentralization is realized by decomposing the robotic team into subgroups based on their missions, %\patrizio{not so clear}
		and then by running a classical planning algorithm.
		Partial knowledge is handled by calling several times  a classical planning algorithm.
		
		Demo video available at: \url{https://youtu.be/TJzC_u2yfzQ}
	\end{abstract}
	
	\section{Introduction}
	Nowadays, most of the planners consider the model of the environment as known and not dynamic~\cite{7139412}. 
However, this is not a real condition of real world scenarios, where only \emph{partial knowledge} can be ensured.
For this reason, our tool is able to compute a plan even when only partial information of the environment is available, as seen in \cite{roy2006planning,du2012robot,diaz2001exploring}.
However, the novelty of our work consists in fuse all this features, exploiting a \emph{decentralized} methodology.
This kind of approaches are not yet studied in detail, due to there are only a few planners managing this issues \cite{guo2015multi}.

\toolName\ provides a planner where a robot application is defined using finite transition systems.
A \emph{planner} is  a software component that receives as input a model of the robotic application and computes  a set of actions (a \emph{plan}) that, if performed, allows the achievement of a desired mission~\cite{latombe2012robot}.
Each robot application contains the robots that conform the team and the mission that they have to achieve.

For this work, we differentiate the aforementioned term of missions between \emph{global missions} and \emph{local missions}.
A \emph{global mission} represents the high-level mission that must be accomplished by the whole team \cite{kloetzer2011multi,loizou2005automated,quottrup2004multi} and that is decomposed into a set of \emph{local missions}\cite{schillinger2016decomposition,guo2015multi,guo2015multi,tumova2016multi}.
Every robot is commanded to achieve a local mission, specified as a LTL property.
As seen in \cite{tumova2016multi}, this collaborative fashion of accomplishing the global mission is performed in a \emph{decentralized} way.
Each robot that is part of a subset of the team computes the solution for its own sub-mission, avoiding the expensive fully centralized planning and making it more robust to local problems.

The present paper is a companion of another submitted work.
Additional information including a running example can be found in~\cite{mapmaker17}.

\textbf{Organization.} 
Section~\ref{sec:approach} describes the \toolName\ approach.
Section~\ref{sec:tool} presents the \toolName\ tool.
Section~\ref{sec:related} introduces different works with a similar scope and position our research.
Section~\ref{sec:conclusion} concludes with final remarks.

\claudio{Take a look at this paper for inspiration of the next sections: "COVER: Change-based Goal Verifier and Reasoner"}
	
	\section{MApMAKER's Overview}
	\label{sec:approach}
	
An overview of \toolName\ is depicted in Fig.~\ref{fig:overview}.
MAPmAKER's planner takes as input the models of the robots (\circled{1}) and of the environment in which they are deployed (\circled{2}) and the mission each robot should achieve (\circled{3}).
Both the models of the robots and their environment may be partial since there can be uncertainty about information contained in these models.
\toolName\ uses the model of the environment and the robots to compute plans that allow the achievement of missions using an appropriate planner.
The implemented planner is able to compute plans that definitely ensure the mission satisfaction, i.e., definitive plans  (\circled{4}), and plans that may ensure property satisfaction since they depend on some partial knowledge present in the models of the robots and the environment  (\circled{5}).
More precisely, a \emph{definitive plan} is a sequence of actions---e.g., move from \emph{a} to \emph{b}---that ensure the satisfaction of the local mission for each robot. 
A \emph{possible plan} is a sequence of actions that may satisfy the local mission due to some unknown information about the model of the robots or the environment in which they are deployed. 
If \toolName\ is not able to find neither a definitive nor a possible plan a message is sent to the user (\circled{6}).
Otherwise, an appropriate component is used to choose between definitive and possible plans (if both are present) or simply chooses the possible plan if no definitive plan is present.
Definitive plans are not present when the only way to satisfy the local mission is based on some unknown information about the model of the robots or the environment in which they are deployed. 
\toolName\ then executes the selected plan (\circled{7}).

As robots perform plans, information about uncertain parts of the model is detected.
\toolName\ updates the  models with the detected information (\circled{8}) and if it detects that a plan is not anymore executable, the planner is re-executed (\circled{9}).


In the following, we provide some additional information about the inputs processed  by \toolName, the planning algorithm, the selection between definitive and possible plans and how models are updated when information about uncertain parts is detected.


\textbf{Models of the robots and their environment}. 
The models of the robots and their environments are provided using a specific form of transition system that allows the specification of uncertain parts; further information might be found in~\cite{menghi2018multi}.
%These models describe the initial positions of the robots, the map describing the environment where robots are deployed, and how robots can move between different locations of the map. 
There is one \texttt{Robotx.m} file for each robot in the global mission.
Each file contains information about the robot, like its id, the atomic prepositions of the LTL formula it may perform, its initial position, services provided by the robot, the id of other robots that must synchronize with it and where the synchronization is performed, etc.
%The model can be generated by passing as arguments the map of the environment, the \emph{possible} map of the environment (same map but containing uncertainty), and the robot initial position to the function \texttt{Robotx} (e.g., \texttt{Robot1}).
This model can be generated by executing the function \texttt{Robotx} (e.g., \texttt{Robot1}).
There is also a \texttt{MissionRobotx.m} file containing a correspondent function for each robot model.
The function encodes the number of actions the robot must perform, whether it is required that other robots must help it to accomplish certain actions and returns an automaton corresponding with a certain formula (the transitions within this file describe the formula).
The environment is defined as a grid where transitions between its conforming cells may or may not be possible.
This information is encoded in the environment model (e.g., \texttt{RealEnvironmentMap.m}).
However, there is always another model of the environment containing partial information (some transitions are now uncertain).
For more technical details, visit the provided repository.


The proposed models embed partial knowledge as follows:\\
%\textbf{Uncertainties supported by \toolName}.
$\bullet$  \emph{Partial knowledge about the actions execution.} 
The execution of certain actions is uncertain, meaning that it is unclear whether an action can be executed.
This type of partial knowledge allows specifying that the transition between two of the cells that conform the grid map of the environment (see Fig.~\ref{fig:outputexample}) can be:
always possible, always impossible (i.e. a wall), not known (i.e. a door between two rooms that can be open or closed).\\
$\bullet$ \emph{Unknown service provisioning.} 
It is unclear  whether a service---i.e.,  ``events of interest associated with execution of certain actions rather than over atomic propositions''~\cite{guo2015multi}---can be provided or not in a specific location. 
For example, it is unclear whether a robot can take a picture of an item in a given map location.
This  uncertainty may be caused for example by the presence of an unexpected object that covers the robot visual in that location.\\
$\bullet$ \emph{Unknown meeting capabilities.} Robots can meet and synchronize in certain locations.
For example, it is unclear whether two robots can exchange a load in a given map location.
This uncertainty may be caused by a collapsing registered in the environment where the robots are deployed.


\begin{figure}[t]
\begin{center}
\includegraphics[width=1\linewidth]{Figures/MAPmAKER.pdf}
\caption{Overview of  MAPmAKER.}
\label{fig:overview}
%\vspace{-.5cm}
\end{center}
\end{figure}




\textbf{Mission specification.}
Each robot is able to perform a complex mission, which is specified using an LTL formula.
This formula specifies how the services must be provided by the robots.
For example, a mission for a robot $r_1$ may require $r_1$ to  periodically load debris on $r_2$.
Thus, in order to allow robot $r_1$ to fulfill its mission, it is necessary that robots $r_1$ and $r_2$ synchronize their behaviours.





\textbf{Planning.} 
The \emph{Planner} uses the models of the robot(s) and the environment in order to compute plans that allow satisfying the missions of the robots.
%\ugh{The planner decomposes the robots within the robotic application}
The planner distributes the robots within the robotic application into subteams ---that we call ``dependency classes"--- based on the mission that each robot has to achieve.
Each dependency class contains a subset of robots that depend on each other for achieving their missions.
After  dependency classes are computed they are considered in isolation regarding the computation of plans that allow robots to satisfy their missions.

To compute a plan for a dependency class the  LTL formulae that are used to describe missions are evaluated on partial models.
Possible and definitive plans are computed by executing a classical planning algorithm twice: once for computing possible plans and once for computing definitive plans.
%The proposed algorithm There exists a subset of LTL formulae, known in literature as self-minimizing formulae



\textbf{Choosing between definitive and possible plans.}
The plan selector component aims at choosing between possible and definitive plans.
%The tool always try to reach the goal performing the lower number of actions.
%This work does not discuss how to choose between
%37 possible and de nitive plans. 
Several policies can be applied to choose between these plans.
Possible plans can be chosen only in cases in which a definitive plan is  not present.
Another policy may choose the plan with the shortest length, or it may consider non-functional aspects of the plans e.g., time to perform certain actions, or likelihood of detecting true or false evidence about partial information. 
In this work we assume  that the planner always chooses the shortest between the possible and the definitive plan.
This policy may, for example, reduce energy consumption, since every action performed by the robots may consume energy. 
%Thus, shorter plans require less energy.

\textbf{Detection of uncertain information.}
As robots perform actions and navigate within the environment, information regarding uncertain services and meeting capabilities can be detected.
Specifically, robots detect whether actions, services, and meeting capabilities are executable, provided, and possible, respectively.
\toolName\ updates the models of the robots and of the environment with the detected information.
Then, if needed, the planning algorithm is triggered and re-executed.
%This information is shared with the rest of the team so it can be take into account for further planning.







	
	\section{MApMAKER in Action}
	\label{sec:tool}
	--Explain the tool in detail, maybe including a scope-- (add figure for the tool? maybe in the previous section?)


A set $R=\{r_1, r_2, r_3 \}$ of robots  is deployed in the environment graphically described in  Fig.~\ref{fig:example1}.
This environment represents a building made by four rooms $L=\{ l_1, l_2, l_3, l_4 \}$, which has been affected by an earthquake.
The environment is further partitioned in cells, each labeled with an identifier in $c_1, c_2, \ldots, c_{30}$.
Robots $r_1$, $r_2$, and $r_3$ are placed in their initial locations.
Each robot is able to move from one cell to another, by performing action $mov$.
The robots are also able to perform the following actions.
Robot $r_1$ is able to load debris of the building by performing action $ld$. 
In Fig.~\ref{fig:example1} the cells in which a robot $r$ can perform an action $\alpha$ are marked with the label $r(\alpha)$.
Robot $r_2$ can wait until another robot loads debris on it by performing action $rd$ and can unload debris by performing one of the two actions $ud1$ and $ud2$. 
Actions $ud1$ and $ud2$ use different actuators.
Specifically, action $ud1$ uses a gripper while action $ud2$ exploits a dump mechanism.
Robot $r_3$ is able to take pictures by performing action $tp$ and send them using a communication network through the execution of action $sp$. 
Symbols $r_1(ld)$, $r_2(rd)$, $r_2(ud1)$, $r_2(ud2)$, $r_3(tp)$, and $r_3(sp)$ are used in Fig.~\ref{fig:example1} to mark the regions where  actions can be executed by the robots, while movement actions are not reported for graphical reasons.
Each action may be associated with a service, which is a high-level functionality provided by the robot when an action is performed.
For example, actions $ld$, $rd$, $tp$, and $sp$  are associated with the services \emph{load\_carrier}, \emph{detect\_load}, \emph{take\_snapshot}, and \emph{send\_info}, respectively.
Actions $ud1$ and $ud2$ are associated with service \emph{unload}.
The labels $L(\pi,\alpha)=\trueval$ below Fig.~\ref{fig:example1} are used to indicate that a service $\pi$ is associated with  action $\alpha$. 
Robots must meet and  synchronously execute actions. 
In this example, robots $r_1$ and  $r_2$ must meet  in cell $c_7$ and synchronously execute actions $ld$ and $rd$, respectively. 
The cells where meeting is requested are marked with rotating arrows marked with the identifiers of the robots that must meet, meaning that, in order to meet, the robots must be on the same cell to meet.


\begin{figure}[!t]
\begin{center}
\includegraphics[width=0.9\linewidth]{Figures/motivatingExample.pdf}
\caption{An example showing the model of the robots and their environment. Plans computed by \toolName\ are represented by trajectories marked with arrows.}
\label{fig:example1}
\end{center}
\end{figure}

The \emph{mission}  the team of robots has to achieve is to check whether toxic chemicals have been released by the container located in $l_4$.
We assume that the mission is specified through a set of \emph{local missions} assigned to each robot of the team and described in Linear Time Temporal Logic  (LTL).
An LTL formula is obtained by composing actions with standard LTL operators: $\Next$ (next), $\Event$  (eventually),  $\Always$ (always) and $\Until$ (until)~\cite{pnueli1977temporal}. 
In our example the mission  can be specified by means of the following local missions: $\phi_1=\Always (\Event ($\emph{load\_carrier}$))$, 
$\phi_2=\Always (\Event($\emph{detect\_ load} $ \wedge \Event ($\emph{unload}$)))$, 
 $\phi_3=\Always ( \Event ($\emph{take\_snapshot} $\wedge \Event ($\emph{send\_info}$)))$, which are assigned to robot $r_1$, $r_2$ and $r_3$, respectively.
The formulae specify that periodically robot $r_1$ loads debris on $r_2$ (by performing action \emph{load\_carrier}), robot $r_2$ receives debris (when action \emph{detect\_ load} occurs)  and brings them to an appropriate unload area (by performing action \emph{unload}), and robot $r_3$ continuously takes pictures (by performing action \emph{take\_snapshot}) and sends them using the communication network (by performing action \emph{send\_info}).
Informally, while $r_3$ continuously takes pictures and sends them using the communication network, $r_1$ and $r_2$ remove debris to allow $r_3$ having a better view on the container.
The pictures allow verifying whether toxic chemicals have been released by the container.




	
	
	\section{Evaluation}
	\label{sec:evaluation}
	To evaluate  \toolName\ we considered the following research questions: \textbf{RQ1}: How does MAPmAKER help planning in partially known environments? \textbf{RQ2}: How does the employed decentralized algorithm help in plan computation?

To answer RQ1 we  had considered a set of existing examples:
one obtained from the RoboCup Logistics League competition~\cite{karrasrobocup} and an apartment of a large residential facility for senior citizens~\cite{map}.
We created a partial robot application starting from the models of the robots and their environment contained in these examples.
We performed different experiments in which we evaluated the impact of partial information about the action execution, services provisioning and meeting capabilities on the planning procedure.
We compare whether computing possible plans actually helps mission achievement.
This is done by comparing our planner with one that is only able to compute definitive plans.
The results can be summarized as follows.
\toolName\ is effective when  a possible plan is selected, and the robot discovers during the plan execution 
that unknown actions, services, and meeting capabilities are executable, provided, and possible, respectively.
%\toolName\ \ugh{is also effective in the cases in which a possible plan  is computed, it can actually be performed and a classical planning algorithm cannot compute a definitive plan.} 
\toolName\ is also effective when this three conditions are given:
\begin{enumerate*}
\item a possible plan is computed;
\item the possible plan can actually be performed; and 
\item a classical planning algorithm cannot compute a definitive plan.
\end{enumerate*}
This situation occurs in the cases in which the only way to fulfill a mission involves some partial information present in the model.
When \toolName\ chooses a definitive plan, it is as effective as a classical planner that is only able to compute definitive plans.
\toolName\ is not effective whenever a possible plan is chosen but  unknown actions, services, and meeting capabilities turned to be not executable, provided, and possible, respectively.
More information about the considered examples, experimental set up, and the obtained results can be found in~\cite{mapmaker17}.

To answer RQ2 we analyzed the advantages of the decentralized procedure provided by \toolName.
We had considered the set of partial models considered in the previous experiments. 
We added an additional robot, i.e., robot $r_3$, which has a mission that can be achieved without collaborating with neither  robot $r1$ nor with robot $r2$. 
We then executed \toolName\ with the decentralized procedure enabled and disabled. 
Then, we compare each performance. 
When \toolName\ was executed with the  decentralized procedure it computed two dependency classes; one containing robots $r_1$ and $r_2$ and one containing robot $r_3$. 
Viceversa, when the decentralized procedure was disabled, \toolName\ 
analyzed a single team containing  robots $r_1$, $r_2$, and $r_3$. 
The results show a drastic improvement in the efficiency of \toolName\ when the decentralized procedure was enabled.
More information can be found in~\cite{mapmaker17}.

%Then, the \emph{Random model generator} generates a certain number (defined by the user) of tests based on the previously explained models.
%Each of this tests is unique since the generator associates a random uncertainty and an initial position of the robotic team to each of them.
%This process is automatically performed for each experiment, since the uncertainty that is checked changes between them (e.g. execution of transitions in \emph{Experiment 1}, services provisioning in \emph{Experiment 2} and meeting capabilities in \emph{Experiment 3}).
%The generation of models must be performed once, although we provide an already working set in our repository and this step can be skipped.
%The already existing models of the environment represent models of real robot applications, i.e., the RoboCup Logistics League competition~\cite{karrasrobocup} and an apartment of a large residential facility for senior citizens~\cite{map}.
%This set is stored in the folder "ReplicationPackage", where the new scenarios must be saved as well.


	
	
	\section{Conclusion}
	\label{sec:conclusion}
	We presented  \toolName, a  decentralized planner for partially known environments.
\toolName\ does not aim to compete with state of the art planners.
It is realized as a proof of concepts to show that models containing partial information can be efficiently handled in by current planners and that decentralized procedures helps in improving performances.
The current implementation relies on a naive implementation of a planner that comes from literature and has been customized within the proposing framework.
However, the theoretical results show that  any planner can be used within \toolName.
 %\toolName\ solves the decentralized planning problem when partial robot applications are analyzed.
Our evaluation showed how  \toolName\ improves planning in cases in which partial information is present.
We also showed that the implemented decentralized procedure improves the performance.


\section*{Acknowledgements}
This work was supported by the EU H2020 Research and Innovation Programme under GA No. 731869 (Co4Robots).

	\balance
	\bibliographystyle{IEEEtran}
	\bibliography{sigproc}
	
\end{document}
