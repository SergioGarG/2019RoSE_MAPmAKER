--Explain the tool in detail, maybe including a scope, give more details of its usage?-- 

Our tool was developed using the MATLAB~\cite{matlab} environment.
The tool is composed by several MATLAB-based scripts that can be executed independently for performing certain functions.
In order to use \toolName it is sufficient to launch the function \emph{mapmaker\_exp} where the \emph{research question}, the \emph{example} or map of the environment and the \emph{experiment} to check are their arguments.
This function enormously simplifies the performance of the tool, but there are other functionalities that can also be exploited explained in the previous section.

In a normal case where the scenarios are already defined, the only components of the architecture depicted in~\ref{fig:overview} that are used are the \emph{Replication Package}, the \emph{Robot \& Environment models with uncertainty}, the \emph{Planner} itself and its inputs and, of course, the \emph{Solutions container}.
The first component represents a folder where the models of the robot and environment with a certain added uncertainty for the selected experiment are stored.
The results achieved and presented in~\cite{mapmaker17} can be replicated using this models.
They are loaded and set as MATLAB workspace variables in the second component.
The planner needs as input the information from this variables and also from the scripts \emph{Algorithms}, \emph{Utils} and \emph{Configuration files}, that are automatically linked.
During the execution of the planner \sergio{image representing the execution of the tool?} the MATLAB interface shows a window with the goals of the current mission. 
Furthermore, after the computation of the plan it shows the movements of the robots and their knowledge of the environment.
When all the robots have reached their goals or they realized that their local mission is not feasible the experiments concludes.
Once the experiment is concluded, the data extracted (as explained in~\ref{sec:approach}) is stored in the solutions container component.

\textbf{Performance}

In~\cite{mapmaker17}  we have shown that our tool is able to perform a plan even in environments full of uncertainty.
Thus, \toolName~ is able to compute a plan where other planners will not.
However, the computation time and the number of actions to be performed by the robot that tries to follow a possible plan and detects a false evidence are normally greatly increased.
This increasing is due to required re-computing that has to be performed when looking for a new possible plan.

For checking its performance the tool was tested in different environments based on real-world scenarios.
The model of the uncertainty of the environment was randomly created and matched with randomly models of the robots (service and synchronization locations and uncertainties).
The performance of tool was evaluated in two Research Questions divided in three experiments (one for each kind of uncertainty defined for this project).
The results are presented and discussed in~\cite{mapmaker17}.

\textbf{Tool Validation}

The \toolName is available at ~\url{https://goo.gl/GY7ZrG}.
There, the following features can be found:
\begin{enumerate}
\item a complete replication package of the two RQs and experiments proposed at~\cite{mapmaker17},
\item a set of videos showing \toolName\ in action computing and solving the scenarios presented at the previous bullet,
\item a brief user guide which defines the functionalities provided by our tool.
\end{enumerate}
