To evaluate  \toolName\ we considered the following research questions: \textbf{RQ1}: How does MAPmAKER help planning in partially known environments? \textbf{RQ2}: How does the employed decentralized algorithm help in plan computation?

To answer \textbf{RQ1} we  had considered a set of existing examples:
one obtained from the RoboCup Logistics League competition~\cite{karrasrobocup} and an apartment of a large residential facility for senior citizens~\cite{map}.
We created a partial robot application starting from these models.


Then, the \emph{Random model generator} generates a certain number (defined by the user) of tests based on the previously explained models.
Each of this tests is unique since the generator associates a random uncertainty and an initial position of the robotic team to each of them.
This process is automatically performed for each experiment, since the uncertainty that is checked changes between them (e.g. execution of transitions in \emph{Experiment 1}, services provisioning in \emph{Experiment 2} and meeting capabilities in \emph{Experiment 3}).
The generation of models must be performed once, although we provide an already working set in our repository and this step can be skipped.
The already existing models of the environment represent models of real robot applications, i.e., the RoboCup Logistics League competition~\cite{karrasrobocup} and an apartment of a large residential facility for senior citizens~\cite{map}.
This set is stored in the folder "ReplicationPackage", where the new scenarios must be saved as well.

