

To evaluate  \toolName\ we formulate a research question,
\textbf{RQ}: Is MAPmAKER able to perform planning in partially known environments?
%\textbf{RQ1}: How does MAPmAKER help planning in partially known environments? \textbf{RQ2}: How does the employed decentralized algorithm help in plan computation?
%The full evaluation might be found in~\cite{menghi2018multi}. 
To answer it, we  had considered the simulated scenario introduced in Sec.~\ref{sec:tool}.
We created a partial robot application starting from the models of the robots and their environment.
We then introduced uncertainty in the three considered dimensions introduced in Sec.~\ref{sec:approach}.
Examples of such uncertainties are whether the system has certain knowledge about the transition through doors (e.g., the one between cells 37 and 38 in Fig.~\ref{fig:runningexample}) or about the provision of services (e.g., \emph{deliver} at cells 24 and 26).
We introduced uncertainty through a random process and created three different scenario configurations based on the same scenario.
We also randomized the initial position of each robot, creating three different sets of initial configurations.
The nine experiments we performed to validate \toolName~consist of the nine possible combinations of the scenario and initial configurations.

The results show that the decentralized algorithm actually helps in improving performances.
The results show that \toolName~is able to compute plans in situations where traditional planners cannot. 
Furthermore, \toolName also improves the performance in terms of plan length in many situations since it may consider more planning options.
In the folder \texttt{ResultsPaperRoSE} \patrizio{which folder?} we provide a set of videos showing the performance of \toolName~in these experiments.
We also provide results, containing computation time, plan length, false and true evidences found by the robots, and ratio between the definitive and possible plans in terms of computation time and plan length.
The evaluation of the underlying algorithms might be found in~\cite{menghi2018multi}. 

%a set of existing examples: one obtained from the RoboCup Logistics League competition~\cite{karrasrobocup} and an apartment of a large residential facility for senior citizens~\cite{map}.
%We created a partial robot application starting from the models of the robots and their environment contained in these examples.
%We performed different experiments in which we evaluated the impact of partial information about the action execution, services provisioning and meeting capabilities on the planning procedure.
%We compare whether computing possible plans actually helps mission achievement.
%
%To answer RQ2 we analyzed the advantages of the decentralized procedure provided by \toolName.
%We had considered the set of partial models considered in the previous experiments. 
%We added an additional robot, i.e., robot $r_3$, which has a mission that can be achieved without collaborating with neither  robot $r1$ nor with robot $r2$. 
%We then executed \toolName\ with the decentralized procedure enabled (computing two different dependency classes based on the collaboration among robots) and disabled (all the robots are part of the same dependency class). 
%Then, we compare each performance. 
%
%We performed three experiments for each scenario, each of them unique since the generator associates a random uncertainty and an initial position of the robotic team to each of them (by using the given \texttt{randomModelGenerator} function).

%To answer RQ1 we  had considered a set of existing examples:
%one obtained from the RoboCup Logistics League competition~\cite{karrasrobocup} and an apartment of a large residential facility for senior citizens~\cite{map}.
%We created a partial robot application starting from the models of the robots and their environment contained in these examples.
%We performed different experiments in which we evaluated the impact of partial information about the action execution, services provisioning and meeting capabilities on the planning procedure.
%We compare whether computing possible plans actually helps mission achievement.
%This is done by comparing our planner with one that is only able to compute definitive plans.
%The results can be summarized as follows.
%\toolName\ is effective when  a possible plan is selected, and the robot discovers during the plan execution 
%that unknown actions, services, and meeting capabilities are executable, provided, and possible, respectively.
%%\toolName\ \ugh{is also effective in the cases in which a possible plan  is computed, it can actually be performed and a classical planning algorithm cannot compute a definitive plan.} 
%\toolName\ is also effective when this three conditions are given:
%\begin{enumerate*}
%\item a possible plan is computed;
%\item the possible plan can actually be performed; and 
%\item a classical planning algorithm cannot compute a definitive plan.
%\end{enumerate*}
%This situation occurs in the cases in which the only way to fulfill a mission involves some partial information present in the model.
%When \toolName\ chooses a definitive plan, it is as effective as a classical planner that is only able to compute definitive plans.
%The computation time and the length of plans are increased whenever  a possible plan is chosen but unknown actions, services, and meeting capabilities turned to be not executable, provided, and possible, respectively.
%More information about the considered examples, experimental set up, and the obtained results can be found in~\cite{menghi2018multi}.
%
%To answer RQ2 we analyzed the advantages of the decentralized procedure provided by \toolName.
%We had considered the set of partial models considered in the previous experiments. 
%We added an additional robot, i.e., robot $r_3$, which has a mission that can be achieved without collaborating with neither  robot $r1$ nor with robot $r2$. 
%We then executed \toolName\ with the decentralized procedure enabled and disabled. 
%Then, we compare each performance. 
%When \toolName\ was executed with the  decentralized procedure it computed two dependency classes; one containing robots $r_1$ and $r_2$ and one containing robot $r_3$. 
%Viceversa, when the decentralized procedure was disabled, \toolName\ 
%analyzed a single team containing  robots $r_1$, $r_2$, and $r_3$. 
%The results show a drastic improvement in the efficiency of \toolName\ when the decentralized procedure was enabled.
%More information can be found in~\cite{menghi2018multi}.

%Then, the \emph{Random model generator} generates a certain number (defined by the user) of tests based on the previously explained models.
%Each of this tests is unique since the generator associates a random uncertainty and an initial position of the robotic team to each of them.
%This process is automatically performed for each experiment, since the uncertainty that is checked changes between them (e.g. execution of transitions in \emph{Experiment 1}, services provisioning in \emph{Experiment 2} and meeting capabilities in \emph{Experiment 3}).
%The generation of models must be performed once, although we provide an already working set in our repository and this step can be skipped.
%The already existing models of the environment represent models of real robot applications, i.e., the RoboCup Logistics League competition~\cite{karrasrobocup} and an apartment of a large residential facility for senior citizens~\cite{map}.
%This set is stored in the folder "ReplicationPackage", where the new scenarios must be saved as well.

