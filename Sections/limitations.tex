\emph{Decentralized solutions.}
--Kind of state of the art, relate the tool with others and make emphasis on the differences --

Decentralized planning problem has been studied for known environments~\cite{schillinger2016decomposition,guo2015multi,tumova2016multi}.
However, planners for partially known environments do not usually employ decentralized solutions~\cite{roy2006planning,du2012robot,diaz2001exploring}. 

\emph{Dealing with partial knowledge in planning.}
Planning in partially known environments is handled in different ways. 
\begin{enumerate*}
\item Several works (e.g.,~\cite{ding2011ltl,kurniawati2011motion,wolff2012robust,du2012robot,Roy2006,chen2012ltl,nikou2017probabilistic,7078886,7139350,narayanan2015task}) consider probabilities within the planning algorithm.
Most of these works  treat partial information by modeling the robotic application using some form of \emph{Markov decision processes} (MDP).
In some of these works~\cite{ding2011ltl,chen2012ltl} transitions of the robots are associated with probabilities which indicate the probability of reaching the destination of the transition given that an action is performed.
In other works~\cite{wolff2012robust}, transition probabilities are not exactly known but are known to belong to a given uncertainty sets.
Finally, several works~\cite{kurniawati2011motion,Roy2006} consider partially observable Markov decision processes.
All these approaches generally generate plans that maximize the worst-case probability of satisfying a mission.
Differently, our work does not consider probabilities.
\item Several works (e.g,~\cite{lahijanian2016iterative,livingston2012backtracking,l2014safety,nie2016searching,7139412}) studied how to change the planned trajectories when unknown obstacles are detected or when obstacles move in a unpredictable way.
In this case, the used underlying model is some sort of \emph{hybrid model}, i.e., models in which finite state machines are combined with differential equations. 
In~\cite{lahijanian2016iterative}, to plan trajectories the authors use a high-level planner that exploits an abstraction of the hybrid system and the mission to compute high-level plans. 
The low-level planner uses the dynamics of the hybrid system and the suggested high-level plans to explore the state space for feasible solutions.
Every time an unknown obstacle is encountered, the high-level planner modifies the coarse high-level plan online by accounting for the geometry of the discovered obstacle. 
Within this framework, \toolName\ can be considered as a high-level planner that is able to use an abstraction of the hybrid system that contains partial information, i.e., encode unknown obstacles.
\item Some  approaches  analyzed how to update plans when new information about known model of a robotic application is detected (e.g.,~\cite{guo2015multi}). 
Differently, in our approach portions of the model of the robotic application are partially known,  partial knowledge is reduced as true and false evidence about partial information is detected.
Other works (e.g.,~\cite{7139310}), aim at detecting how to explore totally unknown environments.
\item 
Plan synthesis is a particular instance of controller synthesis. 
Controller  synthesis (e.g.,~\cite{cassandras2009introduction,D'ippolito:2013:SNE:2430536.2430543}) aims at finding a component, usually indicated as controller or supervisor, that ensures property satisfaction for all the possible system executions.
Differently, plan synthesis aims at finding a single execution, i.e., a plan that ensures property satisfaction.
The controller synthesis  is usually (\cite{kress2009temporal,wongpiromsarn2009receding,chen2012ltl,livingston2012backtracking,guo2013revising}) performed by solving a two player game between robots and their environment.
The goal is to find a strategy the robots can use that allows always winning the game.
Differently, in our case the planning algorithm ensures that there is a way of completing the \emph{single} (possible) plan that satisfies the property of interest. 
\item \toolName\ can be classified on the boundary between reactive synthesis~\cite{chen2012ltl,livingston2012backtracking,thomas2002automata} techniques and iterative planning~\cite{guo2013revising,maly2013iterative}. 
As reactive synthesis techniques, \toolName\ constructs a control strategy that accounts for every possible variation in the environment, but the computed plan does not allow always winning the  \emph{two player game} between the robots and their environment.
As  iterative planning, a new plan is computed on-the-fly when new information is available.
\end{enumerate*}
