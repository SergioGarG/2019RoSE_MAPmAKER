Robotic applications usually rely on a set of robots that are used to perform a set of missions.
Planners are one of the main ingredients that allow robots achieving their missions.
A \emph{planner} is  a software component that receives as input a model of the robotic application and computes  a set of actions---a \emph{plan}--- that, if performed, allows the achievement of a desired mission~\cite{latombe2012robot}.
%Each 

Nowadays, most of the planners consider the model of the environment as known and not dynamic~\cite{7139412}. 
However, this is not a real condition of real world scenarios
In real scenarios the knowledge of the environment cannot be ensured, so only \emph{partial knowledge} is available.
For this reason, there is an increasing need of tools able to compute a plan even when only partial information of the environment is available, as seen in \cite{roy2006planning,du2012robot,diaz2001exploring}.

In this work, we differentiate the term of missions between \emph{global missions} and \emph{local missions}.
A \emph{global mission} represents the high-level mission that must be accomplished by the whole team \cite{kloetzer2011multi,loizou2005automated,quottrup2004multi} and that is decomposed into a set of \emph{local missions}\cite{schillinger2016decomposition,guo2015multi,guo2015multi,tumova2016multi}.
Every robot is commanded to achieve a local mission, specified as a LTL property.
As seen in \cite{tumova2016multi}, this collaborative fashion of accomplishing the global mission is performed in a \emph{decentralized} way.
Each robot that is part of a subset of the team computes the solution for its own sub-mission, avoiding the expensive fully centralized planning and making it more robust to local problems.
\toolName\ splits the given set of robot that conforms the team into classes depending on the local mission that each of them must achieve.

This work presents  \toolName\ (Multi-robot plAnner for PArtially Known EnviRonments), a \emph{novel} \emph{decentralized} planner for partially known environments.
\toolName\ modifies~\cite{tumova2016multi} by supporting partial knowledge.
Proofs about the soundness of \toolName\ can be found in~\cite{mapmaker17}\footnote{This paper is made available for the reviewers   at  \url{https://goo.gl/GY7ZzG}}.
Evaluation is performed by analysing its behaviour on a robot application obtained from the RobotCup Logistics League competition~\cite{karrasrobocup} and on a robotic application working in an apartment of about 80 m , which is part of
a large residential facility for senior citizens~\cite{map}.

\textbf{Organization.} 
Section~\ref{sec:approach} presents an overview of \toolName.
Section~\ref{sec:tool} defines the way of using the \toolName\ tool.
Section~\ref{sec:related} introduces different works with a similar scope and position our research.
Section~\ref{sec:conclusion} concludes with final remarks.

