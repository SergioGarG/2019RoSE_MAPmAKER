%Helping designers to engineer robotic systems is one of the timely application areas of software engineering. 
%Designing %good 
%robotic systems requires solving %several 
%numerous problems~\cite{ljungblad2005designing}, such as the selection of the types of the robot to be employed (e.g., humanoid, robotic arm), the analysis of the required robot properties (e.g., emergence, emotional), the analysis of the place in which the robot operates (e.g., on the bus, in a birthday party), the activity the robot has to perform (e.g., move object, reach a location), and the users it has to serve (e.g., taxi driver, rock star). 
%These aspects are then used by designers in the selection of appropriate planners.

A \emph{planner} is  a software component that receives as input a model of the robotic application and computes  a set of actions (a \emph{plan}) that, if performed, allows the achievement of a desired mission~\cite{latombe2012robot}.
As done in some recent works in the robotics community (see for example~\cite{fainekos2005temporal,kress2007s,kloetzer2008fully,fainekos2009temporal,wongpiromsarn2010receding,bhatia2010sampling,bhatia2010motion}), in this work we assume that 
a robot application is defined using finite transition systems and
each robot of the team  has to achieve a mission,  indicated as \emph{local mission}, that is specified as an LTL property. 
As opposed to more traditional specification means, such as consensus or trajectory tracking in robot control, A-to-B travel in robot motion planning, or STRIPS or PDDL problem formulations in robot task planning, LTL allows us to specify a rich class of temporal goals that include e.g., surveillance, sequencing, safety, or reachability.

%These are common assumptions made in the definition of planners within the robotic domain (see for example~\cite{fainekos2005temporal,kress2007s,kloetzer2008fully,fainekos2009temporal,wongpiromsarn2010receding,bhatia2010sampling,bhatia2010motion}).
Several works studied centralized planners that are able to manage \emph{teams} of robots that collaborate to achieve a certain goal (a global mission)~\cite{kloetzer2011multi,loizou2005automated,quottrup2004multi}.
Others studied how to decompose a global mission into a set of local missions to be achieved by each robot of the team~\cite{schillinger2016decomposition,guo2015multi,guo2015multi,tumova2016multi}. 
These local missions have been recently exploited by \emph{decentralized} planners~\cite{tumova2016multi}, i.e., planners that instead of evaluating the global mission over the whole team of robots, analyze the satisfaction of local missions inside a subset of the team of robots. 
In this way, the problem of finding a collective team behavior is decomposed into sub-problems that avoid the expensive fully centralized planning.

Another aspect that current planners must consider is partial knowledge  about the environment in which the robots should operate.
Partial knowledge in software development has been strongly studied by the  software engineering community.
For example, partial models have been used to support requirement analysis and elicitation~\cite{menghi2017integrating,menghi2017cover,letier2008deriving}, to help designers in producing a model of the system that satisfies a set of desired properties~\cite{uchitel2009synthesis,uchitel2013supporting,famelis2012partial,albarghouthi2012under} and to verify whether  already designed models possess some properties of interest~\cite{menghi2016dealing,bruns1999model,chechik2004multi}.
However, most of the existing planners assume that the environment in which the robots are deployed is known~\cite{7139412}. 
This assumption does not usually hold in real word scenarios~\cite{lahijanian2016iterative}.
In real world applications it is usually the case that only \emph{partial knowledge} about the environment in which the robots are operating is present.
%This occurs, for example, when the robots navigate in environments affected by natural disasters, where the movement between locations or the execution of specific actions may be impossible due to structural collapses, flooding etc.
Several works studied planners that work when only partial information about the environment in which the robots operate is available (e.g.,~\cite{roy2006planning,du2012robot,diaz2001exploring}).
However, %to the best of our knowledge, 
literature considering \emph{decentralized} planners %have not been applied when 
with only partial knowledge about the robot application and temporal logic goals
is rather limited \cite{guo2015multi}.
% is available .


\textbf{Organization.} 
Section~\ref{sec:limitations} introduces robotic applications by highlighting the status of current planners.
Section~\ref{sec:approach} describes the \toolName\ approach.
Section~\ref{sec:tool} presents the \toolName\ tool.
Section~\ref{sec:conclusion} concludes with final remarks.