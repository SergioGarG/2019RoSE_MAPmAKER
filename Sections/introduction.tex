Nowadays, most of the planners consider the model of the environment as known and not dynamic~\cite{7139412}. 
However, this is not a real condition of real world scenarios, where only \emph{partial knowledge} can be ensured \claudio{In my opinion partial knowledge cannot be ensured, partial knowledge is present or available}.
For this reason, our tool  is able to
\claudio{This is still not the moment for discussing the contribution, (the contribution is usually discussed in the last two paragraphs, I would change ``our toll" with ``there is an increasing need of tools able to"}  compute a plan even when only partial information of the environment is available, as seen in \cite{roy2006planning,du2012robot,diaz2001exploring}.
However, the novelty of our work consists in fuse all this features\claudio{which features?}, exploiting a \emph{decentralized} methodology.
This kind of approaches are not yet studied in detail, due to there are only a few planners managing this issues \cite{guo2015multi}.

\toolName\ provides a planner where a robot application is defined using finite transition systems.
A \emph{planner} is  a software component that receives as input a model of the robotic application and computes  a set of actions (a \emph{plan}) that, if performed, allows the achievement of a desired mission~\cite{latombe2012robot}.\claudio{You already mentioned a ``plan" without defining it, probably the plan definition should be before the first use of the term plan.}
Each robot application contains the robots that conform the team and the mission that they have to achieve.

For this work, we differentiate the aforementioned term of missions between \emph{global missions} and \emph{local missions}.
A \emph{global mission} represents the high-level mission that must be accomplished by the whole team \cite{kloetzer2011multi,loizou2005automated,quottrup2004multi} and that is decomposed into a set of \emph{local missions}\cite{schillinger2016decomposition,guo2015multi,guo2015multi,tumova2016multi}.
Every robot is commanded to achieve a local mission, specified as a LTL property.
As seen in \cite{tumova2016multi}, this collaborative fashion of accomplishing the global mission is performed in a \emph{decentralized} way.
Each robot that is part of a subset of the team computes the solution for its own sub-mission, avoiding the expensive fully centralized planning and making it more robust to local problems.

\claudio{Here is the place where the paragraph of MAPmAKER should be located}.
This paper presents \toolName\ and is a companion paper of~\cite{mapmaker17}.
\claudio{\toolName\ bla bla bla}
\claudio{Proofs about the soundness of \toolName\ can be found in~\cite{mapmaker17}\footnote{This paper is made available for the reviewers   at  \url{https://goo.gl/GY7ZzG}.}.
Evaluation is performed by analysing its behaviour on a robot application obtained from the RobotCup Logistics League competition~\cite{} and on a robotic application working in an apartment of about 80 m , which is part of
a large residential facility for senior citizens~\cite{}.}
%Additional information including a running example can be found in~\cite{mapmaker17}.

\claudio{The following paragraph goes in the intro}
This work presents  \toolName\ (Multi-robot plAnner for PArtially Known EnviRonments), a \emph{novel} \emph{decentralized} planner for partially known environments.
\toolName\ modifies~\cite{tumova2016multi} by supporting partial knowledge.
This tool splits the given set of robot that conforms the team into classes depending on the local mission that each of them must achieve.


\textbf{Organization.} 
Section~\ref{sec:approach} describes the \toolName\ approach.
Section~\ref{sec:tool} presents the \toolName\ tool.
Section~\ref{sec:related} introduces different works with a similar scope and position our research.
Section~\ref{sec:conclusion} concludes with final remarks.

