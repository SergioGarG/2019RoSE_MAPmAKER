Nowadays, most of the planners consider the model of the environment as known and not dynamic~\cite{7139412}. 
However, this is not a real condition of real world scenarios, where only \emph{partial knowledge} can be ensured.
For this reason, our tool is able to compute a plan even when only partial information of the environment is available, as seen in \cite{roy2006planning,du2012robot,diaz2001exploring}.
However, the novelty of our work consists in fuse all this features, exploiting a \emph{decentralized} methodology.
This kind of approaches are not yet studied in detail, due to there are only a few planners managing this issues \cite{guo2015multi}.

\toolName\ provides a planner where a robot application is defined using finite transition systems.
A \emph{planner} is  a software component that receives as input a model of the robotic application and computes  a set of actions (a \emph{plan}) that, if performed, allows the achievement of a desired mission~\cite{latombe2012robot}.
Each robot application contains the robots that conform the team and the mission that they have to achieve.

For this work, we differentiate the aforementioned term of missions between \emph{global missions} and \emph{local missions}.
A \emph{global mission} represents the high-level mission that must be accomplished by the whole team \cite{kloetzer2011multi,loizou2005automated,quottrup2004multi} and that is decomposed into a set of \emph{local missions}\cite{schillinger2016decomposition,guo2015multi,guo2015multi,tumova2016multi}.
Every robot is commanded to achieve a local mission, specified as a LTL property.
As seen in \cite{tumova2016multi}, this collaborative fashion of accomplishing the global mission is performed in a \emph{decentralized} way.
Each robot that is part of a subset of the team computes the solution for its own sub-mission, avoiding the expensive fully centralized planning and making it more robust to local problems.

The present paper is a companion of another submitted work.
Additional information including a running example can be found in~\cite{mapmaker17}.

\textbf{Organization.} 
Section~\ref{sec:approach} describes the \toolName\ approach.
Section~\ref{sec:tool} presents the \toolName\ tool.
Section~\ref{sec:related} introduces different works with a similar scope and position our research.
Section~\ref{sec:conclusion} concludes with final remarks.

\claudio{Take a look at this paper for inspiration of the next sections: "COVER: Change-based Goal Verifier and Reasoner"}