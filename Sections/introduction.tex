Robotic applications usually rely on a set of robots that are used to perform missions.
The term mission can refer to a \emph{global mission}, i.e., the high-level mission that must be accomplished by the whole team~\cite{quottrup2004multi} or a \emph{local mission}, i.e., the mission that should be achieved by a single robot, possibly by collaborating with other robots~\cite{tumova2016multi}.
%Every robot is commanded to achieve a local mission, specified as a LTL property.
Planners are one of the main ingredients that allow robots to achieve  missions.
A \emph{planner} is  a software component that receives as input a model of the robotic application and computes  a set of actions---a \emph{plan}---that, if performed, allows the achievement of a desired mission~\cite{latombe2012robot}.
Recent works in robotics have defined robot applications using finite transition systems and some of them define their local missions as a Linear-time Temporal Logic (LTL) property (e.g., \cite{menghi2018multi,guo2015multi,tumova2016multi}).
%\del{LTL enables the enrichment of the specification of temporal goals that are used as input for planners.}

Current robotic applications require planners to address two main challenges: 
\begin{enumerate*}
%\claudio{I am just a bit concern about the term Tractability since Mapmaker does not make the problem tractable. I suggest to remove it and just focus on decentralization.}
\item the planning algorithm should work when (only) partial knowledge about the system---including the robots and their working environment---is present;
\item the planning problem should be solved by decentralized algorithms that helps to reduce the planning overhead.
\end{enumerate*}


%\sergio{in the following paragraphs I cite several works to speak trying to give some background but also related work. Is it enough?}

%Tractability refers to the capability of computer algorithms in solving problems. \claudio{I suggest to delete the previous sentence}
Several works studied centralized planners that are able to manage \emph{teams} of robots that collaborate to achieve a certain goal (a global mission)~\cite{kloetzer2011multi,loizou2005automated,quottrup2004multi}.
However, planning is computationally expensive, especially when the number of robots within the team is large \patrizio{do we have a measure for the size when the problem is becoming difficult with existing planners?} and they need to collaborate to fulfill their local missions.
For this reason, research interest had focused on decomposing a global mission into a set of local missions to be achieved by each robot of the team~\cite{schillinger2016decomposition,guo2015multi,tumova2016multi}. 
These local missions have been recently exploited by \emph{decentralized} planners~\cite{tumova2016multi}, i.e., planners that instead of evaluating the global mission over the whole team of robots, analyze the satisfaction of local missions inside a subset of the team of robots. 
In this way, the problem of finding a collective team behavior is decomposed into sub-problems that avoid the expensive fully centralized planning. 
However, the applicability of these algorithms has never been studied when only partial knowledge about the system is available.



The role of partial knowledge in software development has been strongly studied in literature.
Research has been done on how to consider partial knowledge in requirement analysis and elicitation~\cite{menghi2017integrating,menghi2017cover,letier2008deriving}, in the development of a model of the system that satisfies a set of desired properties~\cite{uchitel2009synthesis,uchitel2013supporting,famelis2012partial,albarghouthi2012under,Bernasconi2017}, and in checking   whether an  already designed model possesses some properties of interest~\cite{menghi2016dealing,bruns1999model,chechik2004multi}.
However, most of the existing planners assume that the environment in which the robots are deployed is known~\cite{7139412}. 
This assumption does not usually hold in real-word scenarios~\cite{lahijanian2016iterative},  where, for example,  the robots navigate in environments affected by natural disasters, where the movement between locations or the execution of specific actions may be impossible due to structural collapses, flooding, etc.
The planners that consider  partial information about the environment in which the robots operate (e.g.,~\cite{roy2006planning,du2012robot,diaz2001exploring}) usually rely on probabilistic algorithms and are not  \emph{decentralized}.
%However, %to the best of our knowledge, 
%literature considering \emph{decentralized} planners %have not been applied when 
%with only partial knowledge about the robot application and temporal logic goals
%is rather limited \cite{guo2015multi}.

%Nowadays, most of the planners consider the model of the environment as known and not dynamic~\cite{7139412}. 
%However, this is not a real condition of real world scenarios
%In real scenarios the knowledge of the environment cannot be ensured, so only \emph{partial knowledge} is available.
%For this reason, there is an increasing need of tools able to compute a plan even when only partial information of the environment is available, as seen in \cite{roy2006planning,du2012robot,diaz2001exploring}.


%As seen in \cite{tumova2016multi}, this collaborative fashion of accomplishing the global mission is performed in a \emph{decentralized} way.
%Each robot that is part of a subset of the team computes the solution for its own sub-mission, avoiding the expensive %fully centralized planning and making it more robust to local problems.
%\toolName\ splits the given set of robot that conforms the team into classes depending on the local mission that each of them must achieve.

This work presents  \toolName: a Multi-robot plAnner for PArtially Known EnviRonments.
 \toolName\ provides a  \emph{decentralized} planning solution that works in \emph{partially known} environments.
%\toolName\ modifies~\cite{tumova2016multi} by supporting partial knowledge.
Decentralization is realized by decomposing the robotic team into subteams based on their missions, and then by running a classical planning algorithm.
Partial knowledge is handled by calling twice a classical planning algorithm.
The theory that supports \toolName\ including proofs of correctness, a detailed description of the modelling formalisms and the verification procedures can be found in~\cite{menghi2018multi}.
In this paper we present the implementation of \toolName and we show how planners can deal with the previously stated features. In this sense the contribution is more a proof of concept than a tool ready to be used for real-world scenarios.
\toolName~builds upon the planner proposed by Tumova et al.~\cite{tumova2016multi}.
\toolName\ is evaluated by analysing its behaviour on a robot application simulating a hospital environment with the presence of random uncertainty.
%obtained from the RobotCup Logistics League competition~\cite{karrasrobocup} and on a robotic application working in an apartment of about 80 m$^2$~\cite{map}.
%, which is part ofa large residential facility for senior citizens.
\toolName\  together with 
\begin{enumerate*}
\item a complete replication package,
\item a set of videos showing \toolName\ in action computing and solving the scenario presented at the previous bullet, and
\item a brief user guide that defines the functionalities provided by our tool
\end{enumerate*}
 are available at our Github repository~\cite{repo}. %~\url{https://github.com/claudiomenghi/MAPmAKER/}. 

%This paper is organized as follows. 
%Section~\ref{sec:approach} presents an overview of \toolName.
%Section~\ref{sec:tool} describes how  \toolName\ can be used.
%Section~\ref{sec:evaluation} evaluates  \toolName.
%Section~\ref{sec:conclusion} concludes with final remarks.

