We presented  \toolName, a  decentralized planner for partially known environments.
The theoretical results show that any planner can be used within \toolName. %\toolName\ does not aim to compete with state of the art planners.
It is realized as a proof of concepts to show that (i) models containing partial information can be efficiently handled in by current planners and that (ii) decentralized procedures help in improving performances.
The current implementation relies on a naive implementation of a planner that comes from literature and has been customized within the proposing framework.
 %\toolName\ solves the decentralized planning problem when partial robot applications are analyzed.
Our evaluation showed how  \toolName\ improves planning in cases in which partial information is present.
We also showed that the implemented decentralized procedure improves the performance of the planning algorithm.

As future work we will experiment with more complex scenarios and with real robots.
To do so, we will use more efficient planners to speed up the computation.
Other work will include the study of appropriate policies to select between definitive and possible plans.
